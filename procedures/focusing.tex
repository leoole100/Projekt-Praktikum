\section*{Focusing}

If the system is already roughly aligned, you may begin at step 3.

\begin{enumerate}
  \item \textbf{Initial setup with fiber lamp.}  
  Remove the sample and place a fiber-coupled lamp at the sample position. This will help align and focus the optics using visible light.

  \item \textbf{Align the beam.}  
  Adjust the lamp position that the beam appears circular and sharply focused at the spectrometer fiber port. This ensures the optics are approximately aligned.

  \item \textbf{Connect and optimize with spectrometer.}  
  Connect the spectrometer to the spectrometer port and move it around to maximize the signal. Use the micrometer screws on the sample stage for fine adjustments.  
  The focused spot size is given by
  \(
   2.44 f \lambda /D \approx \SI{2}{\micro\meter}
  \)
  which is slightly larger than the \SI{1.5}{\micro\meter} fiber core, so precise alignment is important, but there should be some play, where the signal does not change.

  \item \textbf{Swap back in the sample.}  
  Once the lamp is well-focused, remove it and place the actual sample at the same position. Connect the lamp output to the spectrometer port to illuminate the sample.

  \item \textbf{Fine-focus the sample.}  
  Adjust the sample position to minimize the visible spot size. The spot on the sample should now be even smaller and sharply defined.

  \item \textbf{Focus the infrared laser.}  
  Turn on the infrared laser and use an IR detection card to locate and focus the beam onto the sample.

  \item \textbf{Optimize laser alignment.}  
  Disconnect the lamp and monitor the reflected laser signal on the spectrometer. Fine-tune the laser focus to maximize this signal.
\end{enumerate}

